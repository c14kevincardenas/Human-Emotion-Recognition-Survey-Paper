Innately, humans yearn for social interaction and it is of paramount importance to everyone as a means to interact with people, conduct business, and at its core it is how we discover our own identity, our purpose in life, and our place in the world. The underlying aspect that drives social interaction are human emotions. Humans are arguably one of the most complex and intricate beings in the world and more so are the plethora of types and intensities of emotions humans display. Human Emotion Recognition (HER) is being heavily researched, however as a result of the complexity of human emotions it is challenging to understand which techniques are the most successful to truly recognize human emotions. One objective of this paper is to summarize the four primary modalities of human emotion recognition including facial expression recognition, posture and gesture recognition, physiological signal based emotion recognition, and speech emotion recognition (including tone of voice and spoken words). However, as elaborate beings humans do not simply express emotion solely through one of these modalities and often use a combination to express and recognize emotion. This article surveys state-of-the-art techniques and databases for conducting a multimodal approach to recognizing human emotion.
